\documentclass[UTF8]{ctexart}
\usepackage{geometry, CJKutf8}
\geometry{margin=1.5cm, vmargin={0pt,1cm}}
\setlength{\topmargin}{-1cm}
\setlength{\paperheight}{29.7cm}
\setlength{\textheight}{25.3cm}

% useful packages.
\usepackage{amsfonts}
\usepackage{amsmath}
\usepackage{amssymb}
\usepackage{amsthm}
\usepackage{enumerate}
\usepackage{graphicx}
\usepackage{multicol}
\usepackage{fancyhdr}
\usepackage{layout}
\usepackage{listings,xcolor}
\usepackage{float, caption}

\lstset{
    basicstyle=\ttfamily, basewidth=0.5em
}

% some common command
\newcommand{\dif}{\mathrm{d}}
\newcommand{\avg}[1]{\left\langle#1 \right\rangle}
\newcommand{\difFrac}[2]{\frac{\dif#1}{\dif#2}}
\newcommand{\pdfFrac}[2]{\frac{\partial#1}{\partial#2}}
\newcommand{\OFL}{\mathrm{OFL}}
\newcommand{\UFL}{\mathrm{UFL}}
\newcommand{\fl}{\mathrm{fl}}
\newcommand{\op}{\odot}
\newcommand{\Eabs}{E_{\mathrm{abs}}}
\newcommand{\Erel}{E_{\mathrm{rel}}}

\begin{document}

\pagestyle{fancy}
\fancyhead{}
\lhead{周妍祯, 3210103986}
\chead{数据结构与算法第五次作业}
\rhead{Oct.31th, 2024}

\section{remove函数的修改}
原本的 remove 函数在处理删除的节点有 2 个子节点的情况时,先查找右子树中的最小元素赋给当前节点,
然后再查找一次最小元素进行删除,一共进行了 2 次搜索而且进行了值的复制,而实际上只需要 1 次查找和替换就可以完成。

因此,一种更加高效的实现方法是,在找到右子树最小元素的时候就进行删除,并将该节点的指针直接返回,
然后对该节点的 left 和 right 指针进行修改即可。

首先,我实现了 BinaryNode *detachMin(BinaryNode *\&t) 函数,
该函数的作用是查找以 t 为根的子树中的最小节点,返回这个节点,并从原子树中删除这个节点。
具体函数实现如下:
\begin{lstlisting}[language=c++, breaklines=true, keywordstyle=\color{blue!70}, commentstyle=\color{red!50!green!50!blue!50}, frame=shadowbox, rulesepcolor=\color{red!20!green!20!blue!20}]
    BinaryNode *detachMin(BinaryNode *&t) {
        if (t == nullptr)
            return nullptr;
        if (t->left != nullptr)
            return detachMin(t->left);
        else {
            BinaryNode *minNode = t;
            t = t->right;
            return minNode;
        }
    }
\end{lstlisting}

在 detachMin 函数中,如果查找的树是空树则返回 nullptr,如果不是空树,则判断左子节点是否存在,
若左子节点存在,则递归 detachMin(t->left) 直到找到最小值,通过 t = t->right 实现从树中删除这个节点,
然后直接返回找到的节点。

在 detachMin 函数的基础上,我对 remove 函数进行了如下修改:
\begin{lstlisting}[language=c++, breaklines=true, keywordstyle=\color{blue!70}, commentstyle=\color{red!50!green!50!blue!50}, frame=shadowbox, rulesepcolor=\color{red!20!green!20!blue!20}]
    void remove(const Comparable &x, BinaryNode * &t) {
        if (t == nullptr) {
            return; 
        }
        if (x < t->element) {
            remove(x, t->left);
        } else if (x > t->element) {
            remove(x, t->right);
        } 
        else if (t->left != nullptr && t->right != nullptr) { 
            BinaryNode *oldNode = t;
            t = detachMin(t->right); 
            t->left = oldNode->left; 
            if(oldNode->right == t) 
                t->right = nullptr; 
            else
                t->right = oldNode->right;
            delete oldNode;
        } else {
            BinaryNode *oldNode = t;
            t = (t->left != nullptr) ? t->left : t->right;
            delete oldNode;
        }
    }
\end{lstlisting}

主要改动在第 2 个 else if 分支内,因为这里进行的是节点的替换,
因此在将查找到的节点替换前,要保存当前节点信息,以便后续释放空间,我使用 oldNode 来保存当前节点 t. 
之后调用 detachMin(t->right) 找到右子树的最小节点,直接替换 t. 
考虑到替换的节点有可能正好是右子节点,因此需要判断 oldNode->right == t 是否成立,如果成立,则直接将 nullptr 赋给 t->right,
如果不成立,则将原本节点的右子节点赋给 t->right. 替换的节点不可能是左子节点,因此直接将原本节点的左子节点赋给 t->left 即可。
最后还需要释放 oldNode. 


\section{测试程序的设计思路}
我修改了老师提供的 testBinarySearchTree 函数来测试 remove 函数的功能,首先初始化一个二叉查找树 bst,然后往里面依次插入
10, 5, 15, 3, 7, 12, 18. 这个二叉树正好是满的,同时打印二叉树将会得到递增的序列 3, 5, 7, 10, 12, 15, 18. 
接下来是测试 move 函数,我分成了 3 部分进行测试:
\begin{enumerate}
    \item 为了测试叶子节点(即没有 child 的节点)的删除,我调用了 bst.remove(7) 来删除叶子节点 7,删除后节点 5 只有 1 个 child,其值为 3;
    \item 为了测试只有 1 个 child 的节点的删除,我调用了 bst.remove(5) 来删除节点 5,这时节点 5 被节点 3 替代,且 节点 3 没有 child;
    \item 为了测试有 2 个 child 的节点的删除,我先插入了 1 和 4,这两个新插入的节点将放置在节点 3 的两侧,作为它的左右子节点,
然后调用 bst.remove(3) 来删除节点 3,由于此时节点 3 的右子树只有 1 个元素 4,因此将由节点 4 替代节点 3,同时节点 4 不再有右子节点;
最后我调用 bst.remove(10) 来删除根节点 10,此时右子树的最小元素不再是根节点的右子节点,而是节点 15 的左子节点 12,
故节点 12 将替代根节点,同时节点 15 仍是根节点的右子节点,但节点 15 的左子节点为空。
\end{enumerate}

在每次调用 remove 函数后我都将结果打印出来,如果所有结果均是顺序排列的则表示测试通过。


\section{测试的结果}

测试结果输出如下:
\begin{lstlisting}[language=c++, breaklines=true, frame=shadowbox, rulesepcolor=\color{red!20!green!20!blue!20}]
Initial Tree:
3
5
7
10
12
15
18
Tree after removing 7:
3
5
10
12
15
18
Tree after removing 5:
3
10
12
15
18
Tree after removing 3:
1
4
10
12
15
18
Tree after removing 10:
1
4
12
15
18
\end{lstlisting}

测试结果一切正常。

我用 valgrind 进行测试,发现没有发生内存泄露。


\end{document}

%%% Local Variables: 
%%% mode: latex
%%% TeX-master: t
%%% End: 
