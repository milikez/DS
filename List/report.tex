\documentclass[UTF8]{ctexart}
\usepackage{geometry, CJKutf8}
\geometry{margin=1.5cm, vmargin={0pt,1cm}}
\setlength{\topmargin}{-1cm}
\setlength{\paperheight}{29.7cm}
\setlength{\textheight}{25.3cm}

% useful packages.
\usepackage{amsfonts}
\usepackage{amsmath}
\usepackage{amssymb}
\usepackage{amsthm}
\usepackage{enumerate}
\usepackage{graphicx}
\usepackage{multicol}
\usepackage{fancyhdr}
\usepackage{layout}
\usepackage{listings,xcolor}
\usepackage{float, caption}

\lstset{
    basicstyle=\ttfamily, basewidth=0.5em
}

% some common command
\newcommand{\dif}{\mathrm{d}}
\newcommand{\avg}[1]{\left\langle#1 \right\rangle}
\newcommand{\difFrac}[2]{\frac{\dif#1}{\dif#2}}
\newcommand{\pdfFrac}[2]{\frac{\partial#1}{\partial#2}}
\newcommand{\OFL}{\mathrm{OFL}}
\newcommand{\UFL}{\mathrm{UFL}}
\newcommand{\fl}{\mathrm{fl}}
\newcommand{\op}{\odot}
\newcommand{\Eabs}{E_{\mathrm{abs}}}
\newcommand{\Erel}{E_{\mathrm{rel}}}

\begin{document}

\pagestyle{fancy}
\fancyhead{}
\lhead{周妍祯, 3210103986}
\chead{数据结构与算法第四次作业}
\rhead{Oct.18th, 2024}

\section{测试程序的设计思路}
我在 main 函数中调用了多个函数来测试不同的功能,如果这些函数均能正常运行,
则会在最后输出 "All tests passed!". 接下来我将依次介绍每一个函数实现了什么。

\subsection{测试默认构造函数 -- testDefaultConstructor}
\begin{lstlisting}[language=c++, breaklines=true, keywordstyle=\color{blue!70}, commentstyle=\color{red!50!green!50!blue!50}, frame=shadowbox, rulesepcolor=\color{red!20!green!20!blue!20}]
    void testDefaultConstructor() {
        List<int> list;
        assert(list.size() == 0);
        assert(list.empty());
        list.printList();   // expect empty line
    }
\end{lstlisting}

我初始化了一个空链表 list,不进行任何的赋值,利用 `assert` 来测试该列表的size是否为0,
以及该列表是否为空,最后打印出这个列表,预期得到一个空行。

\subsection{测试初始化列表构造函数 -- testInitializerListConstructor}
\begin{lstlisting}[language=c++, breaklines=true, keywordstyle=\color{blue!70}, commentstyle=\color{red!50!green!50!blue!50}, frame=shadowbox, rulesepcolor=\color{red!20!green!20!blue!20}]
    void testInitializerListConstructor() {
        List<int> list = {1, 2, 3, 4, 5};
        assert(list.size() == 5);
        assert(*(list.begin()) == 1 && *(--list.end()) == 5);
        list.printList();   // expect: 1 2 3 4 5
    }
\end{lstlisting}

我利用初始化列表构造了一个列表 list,其中放了 5 个值(1 2 3 4 5),
利用 `assert` 来测试该列表的 size 是否为 5,以及该列表的第一个元素和最后一个元素是否为 1 和 5,最后打印出这个列表。

\subsection{测试拷贝构造函数 -- testCopyConstructor}
\begin{lstlisting}[language=c++, breaklines=true, keywordstyle=\color{blue!70}, commentstyle=\color{red!50!green!50!blue!50}, frame=shadowbox, rulesepcolor=\color{red!20!green!20!blue!20}]
    void testCopyConstructor() {
        List<int> list1 = {1, 2, 3, 4, 5};
        List<int> list2(list1);
        assert(list2.size() == 5);
        assert(*list2.begin() == 1 && *(--list2.end()) == 5);
        list1.printList();  // expect: 1 2 3 4 5
        list2.printList();  // expect: 1 2 3 4 5
    }
\end{lstlisting}

我首先用初始化列表构造了 list1,其中存放了 5 个值(1 2 3 4 5),然后使用拷贝构造函数将 list1 拷贝到 list2 中。
为了测试两者是否相同,我使用了 `assert` 来测试 list2 的 size 是否为 5,以及第一个元素和最后一个元素是否为 1 和 5。
最后将 list1 和 list2 都打印出来,看是否相等。

\subsection{测试移动构造函数 -- testMoveConstructor}
\begin{lstlisting}[language=c++, breaklines=true, keywordstyle=\color{blue!70}, commentstyle=\color{red!50!green!50!blue!50}, frame=shadowbox, rulesepcolor=\color{red!20!green!20!blue!20}]
    void testMoveConstructor() {
        List<std::string> list1 = {"one", "two", "three"};
        List<std::string> list2(std::move(list1));
        assert(list1.size() == 0);
        assert(list2.size() == 3);
        assert(*list2.begin() == "one" && *(--list2.end()) == "three");
        list2.printList();  // expect: one two three
    }
\end{lstlisting}

我首先用初始化列表构造了 list1,其中存放了3个值("one" "two" "three"),然后使用移动构造函数将 list1 中的内容全部移动到 list2 中。
为了测试两者是否相同,使用 `assert` 来测试list2的size是否为3,以及第一个元素和最后一个元素是否为"one"和"three"。
最后我只打印了 list2,移动后尝试访问 list1 将导致程序崩溃,因为 head 和 tail 都被指向 nullptr 而不是一个有效的节点。

\subsection{测试赋值运算符 -- testAssignmentOperator}
\begin{lstlisting}[language=c++, breaklines=true, keywordstyle=\color{blue!70}, commentstyle=\color{red!50!green!50!blue!50}, frame=shadowbox, rulesepcolor=\color{red!20!green!20!blue!20}]
    void testAssignmentOperator() {
        List<int> list1 = {1, 2, 3, 4, 5};
        List<int> list2;
        list2 = list1;
        assert(list2.size() == 5);
        assert(*list2.begin() == 1 && *(--list2.end()) == 5);
        list1.printList();  // expect: 1 2 3 4 5
        list2.printList();  // expect: 1 2 3 4 5
    }
\end{lstlisting}

我首先用初始化列表构造了 list1,其中存放了 5 个值(1 2 3 4 5),然后初始化 list2,不进行任何的赋值,之后使用赋值运算符将 list1 拷贝到 list2 中。
为了测试两者是否相同,我使用了 `assert` 来测试 list2 的 size 是否为 5,以及第一个元素和最后一个元素是否为 1 和 5。
最后将 list1 和 list2 都打印出来,看是否相等。

\subsection{测试移动赋值运算符 -- testMoveAssignmentOperator}
\begin{lstlisting}[language=c++, breaklines=true, keywordstyle=\color{blue!70}, commentstyle=\color{red!50!green!50!blue!50}, frame=shadowbox, rulesepcolor=\color{red!20!green!20!blue!20}]
    void testMoveAssignmentOperator() {
        List<std::string> list1 = {"one", "two", "three"};
        List<std::string> list2;
        list2 = std::move(list1);
        assert(list1.size() == 0);
        assert(list2.size() == 3);
        assert(*list2.begin() == "one" && *(--list2.end()) == "three");
        list2.printList();  // expect: one two three
    }
\end{lstlisting}

我首先用初始化列表构造了 list1,其中存放了 3 个值("one" "two" "three"),然后初始化 list2,不进行任何的赋值,之后使用移动赋值运算符将 list1 中的内容移动到到 list2 中。
为了测试移动操作是否正确,我使用了 `assert` 来测试 list1 和 list2 的 size 是否分别为 0 和 3,以及 list2 第一个元素和最后一个元素是否为 "one" 和 "three"。
同样的,我最后只将 list2 打印出来,因为移动后 list1 的 head 和 tail 都被指向 nullptr 而不是一个有效的节点。

\subsection{测试插入和删除函数 -- testInsertAndErase}
\begin{lstlisting}[language=c++, breaklines=true, keywordstyle=\color{blue!70}, commentstyle=\color{red!50!green!50!blue!50}, frame=shadowbox, rulesepcolor=\color{red!20!green!20!blue!20}]
    void testInsertAndErase() {
        List<int> list = {4, 5, 6};
        
        // 插入元素
        list.insert(list.end(), 1);
        list.insert(list.end(), 2);
        list.insert(list.end(), 1 + 2); // 右值
        assert(list.size() == 6);
        assert(*list.begin() == 4 && *(--list.end()) == 3);
        list.printList();   // expect: 4 5 6 1 2 3
    
        // 删除元素
        list.erase(list.begin());
        list.erase(list.begin());
        list.erase(--list.end());
        assert(list.size() == 3);
        assert(*list.begin() == 6 && *(--list.end()) == 2);
        list.printList();   // expect: 6 1 2
    
        // 再次插入元素
        list.insert(list.begin(), 0 + 0);   // 右值
        list.insert(++list.begin(), 7);
        assert(list.size() == 5);
        assert(*list.begin() == 0 && *(++list.begin()) == 7);
        list.printList();   // expect: 0 7 6 1 2
    
        // 删除指定范围的数据节点
        list.erase(++list.begin(), --list.end());
        assert(list.size() == 2);
        assert(*list.begin() == 0 && *(--list.end()) == 2);
        list.printList();   // expect: 0 2
    
        // 清除所有元素
        list.clear();
        assert(list.size() == 0);
        list.printList();   // expect empty line
    }
\end{lstlisting}

我首先用初始化列表构造了 list,其中存放了 3 个值(4 5 6)。

\begin{enumerate}
    \item 测试 insert 函数在尾部插入数据:使用 insert 函数在 list 尾部依次插入 1 2 1+2,这里的 1 2 是左值,而 1+2 是右值,
分别调用了 `iterator insert(iterator itr, const Object \&x)` 和 `iterator insert(iterator itr, Object \&\&x)`。
为了测试插入操作是否正确,我使用了 `assert` 来测试 list 的 size 是否为 6,以及第一个元素和最后一个元素是否为 4 和 3。
最后打印出这个列表。

    \item 测试 erase 函数:使用 erase 函数删除 list 的前 2 个节点数据和最后 1 个节点数据,调用了 `iterator erase(iterator itr)` 函数。
为了测试删除操作是否正确,我使用了 `assert` 来测试 list 的 size 是否为 3,以及第一个元素和最后一个元素是否为 6 和 2。
最后打印出这个列表。

    \item 测试 insert 函数在头部插入数据: 使用 insert 函数在 list 头部插入 0+0,之后在 list 第 2 个位置处插入 7. 
为了测试插入操作是否正确,我使用了 `assert` 来测试 list 的 size 是否为 5,以及第一个元素和第二个元素是否为 0 和 7。
最后打印出这个列表。

    \item 测试 erase 删除指定范围:使用 erase 函数删除第二个元素至倒数第二个元素范围内的数据,最后应只剩下第一个和最后一个元素。
这里调用了 `iterator erase(iterator from, iterator to)` 函数。
为了测试删除操作是否正确,我使用了 `assert` 来测试 list 的 size 是否为 2,以及第一个元素和最后一个元素是否为 0 和 2。
最后打印出这个列表。

    \item 测试 clear 函数:使用 clear 函数清楚 list 的所有元素,这里调用了 `void clear()` 函数。
为了测试清除操作是否正确,我使用了 `assert` 来测试 list 的 size 是否为 0,
最后打印出这个列表,预期得到一个空行。
\end{enumerate}

\subsection{测试push和pop函数 -- testPushAndPop}
\begin{lstlisting}[language=c++, breaklines=true, keywordstyle=\color{blue!70}, commentstyle=\color{red!50!green!50!blue!50}, frame=shadowbox, rulesepcolor=\color{red!20!green!20!blue!20}]
    void testPushAndPop() {
        List<float> list = {0.1, 0.2, 0.3, 0.4};
    
        // 插入数据到list头部
        list.push_front(-0.1);
        list.push_front(1.0 - 1.2); // 右值
        assert(list.size() == 6);
        list.printList();   // expect: -0.2 -0.1 0.1 0.2 0.3 0.4
    
        // 插入数据到list尾部
        list.push_back(0.5);
        list.push_back(0.3 + 0.3);  // 右值
        assert(list.size() == 8);
        list.printList();   // expect: -0.2 -0.1 0.1 0.2 0.3 0.4 0.5 0.6
    
        // 删除第1个节点的数据
        list.pop_front();
        list.pop_front();
        assert(list.size() == 6);
        list.printList();   // expect: 0.1 0.2 0.3 0.4 0.5 0.6
    
        // 删除最后1个节点的数据
        list.pop_back();
        assert(list.size() == 5);
        list.printList();   // expect: 0.1 0.2 0.3 0.4 0.5
    }
\end{lstlisting}

我首先用初始化列表构造了 list,其中存放了 4 个值(0.1 0.2 0.3 0.4)。

\begin{enumerate}
    \item 测试 push\_front 函数:使用 push\_front 函数在 list 头部插入 -0.1 和 1.0 - 1.2,这里的 -0.1 是左值,而 1.0 - 1.2 是右值,
分别调用了 `void push\_front(const Object \&x)` 和 `void push\_front(Object \&\&x)`。
为了测试插入操作是否正确,我使用了 `assert` 来测试 list 的 size 是否为 6,并打印出这个列表。

    \item 测试 push\_back 函数:使用 push\_back 函数在 list 尾部插入 0.5 和 0.3 + 0.3,这里的 0.5 是左值,而 0.3 + 0.3 是右值,
分别调用了 `void push\_back(const Object \&x)` 和 `void push\_back(Object \&\&x)`。
为了测试插入操作是否正确,我使用了 `assert` 来测试 list 的 size 是否为 8,并打印出这个列表。

    \item 测试 pop\_front 函数:使用 pop\_front 函数删除 list 的前 2 个节点数据,调用了 `void pop\_front()` 函数。
为了测试删除操作是否正确,我使用了 `assert` 来测试 list 的 size 是否为 6,并打印出这个列表。

    \item 测试 pop\_back 函数:使用 pop\_back 函数删除 list 的最后一个节点数据,调用了 `void pop\_back()` 函数。
为了测试删除操作是否正确,我使用了 `assert` 来测试 list 的 size 是否为 5,并打印出这个列表。
\end{enumerate}

\subsection{测试迭代器 -- testIterators}
\begin{lstlisting}[language=c++, breaklines=true, keywordstyle=\color{blue!70}, commentstyle=\color{red!50!green!50!blue!50}, frame=shadowbox, rulesepcolor=\color{red!20!green!20!blue!20}]
    void testIterators() {
        List<int> list = {1, 2, 3, 4, 5};
    
        // 测试 iterator
        auto it = list.begin();
        auto cit = list.begin();
        assert(it == cit);
        assert(*it == 1);
        assert(*cit == 1);
        ++it;
        ++cit;
        assert(*it == 2);
        assert(*cit == 2);
        --it;
        --cit;
        assert(it == cit);
        assert(*it == 1);
        assert(*cit == 1);
    
        // 测试 const_iterator
        // 使用const_iterator遍历List
        for (List<int>::const_iterator c_it = list.begin(); c_it != list.end(); ++c_it) {
            std::cout << *c_it << " "; 
        }
        std::cout << std::endl;     // expect: 1 2 3 4 5
        
        // 使用const_iterator验证特定值
        List<int>::const_iterator c_cit = --list.end();
        for (int i = 5; i > 0; --i, --c_cit) {
            assert(*c_cit == i); // 再次使用operator*来获取当前节点的数据
        }
    
        // 测试返回值为 const 的情况
        const List<int>& constList = list;
        assert(constList.front() == 1);
        assert(constList.back() == 5);
    
    }
\end{lstlisting}

在这部分测试中,我首先构造了一个包含整数 1 到 5 的链表 list。

\begin{enumerate}
    \item 测试 iterator 的功能:通过创建一个 iterator 对象 it 和一个 const\_iterator 对象 cit,
我验证了它们是否可以正确访问各个节点的数据。
我使用了 `operator*` 来获取当前节点的数据,并使用 `operator++` 和 `operator--` 来前后移动迭代器的位置,
这里分别测试了前置递增和递减、后置递增和递减。
我还测试了 `operator==` 来判断两个迭代器是否相等。

    \item 测试 const\_iterator 的功能:通过使用 const\_iterator 遍历链表并打印每个节点的数据,
我验证了 const\_iterator 是否可以正确地读取链表中的数据。此外,我还通过递减 const\_iterator 来访问链表的末尾元素,
并使用 `operator*` 获取节点数据进行验证。

    \item 测试 List 被声明为 const 的情况:测试其是否可以正常访问其元素。
通过 `constList.front()` 和 `constList.back()` 的调用,我验证了 const iterator 是否能够返回首尾元素的值。
这里在内部分别调用了 `const\_iterator begin() const` 和 `const\_iterator end() const`。
\end{enumerate}

\subsection{测试边界条件 -- testBoundaryConditions}
\begin{lstlisting}[language=c++, breaklines=true, keywordstyle=\color{blue!70}, commentstyle=\color{red!50!green!50!blue!50}, frame=shadowbox, rulesepcolor=\color{red!20!green!20!blue!20}]
    void testBoundaryConditions() {
        List<int> list;
        assert(list.size() == 0);
        assert(list.empty());
        list.push_back(1);
        assert(list.size() == 1);
        assert(!list.empty());
        list.push_back(2);
        list.push_front(0);
        assert(list.size() == 3);
        assert(list.front() == 0);
        assert(list.back() == 2);
        list.pop_front();
        list.pop_back();
        assert(list.size() == 1);
        assert(list.front() == 1);
        assert(list.back() == 1);
    }
\end{lstlisting}

在这部分测试中,我构造了一个空的链表 list,并进行了一系列的边界条件测试,以确保链表的功能能够正常实现。

首先,我验证了空链表的 size 是否为 0,以及 empty() 函数是否正确返回 true。接着,我向链表中添加了一个元素,
并检查 size 是否正确更新为 1,empty() 函数是否返回 false。
然后,我继续向链表中添加更多元素,并使用 push\_front 函数在头部插入元素,
验证链表的 size 是否正确增加,以及 front() 和 back() 函数是否返回正确的首尾元素值。
最后,我测试了 pop\_front 和 pop\_back 函数,它们分别删除链表的首尾元素。
我验证了这些操作后链表的 size 是否正确减少,以及 front() 和 back() 函数是否返回剩余元素的正确值。


\section{测试的结果}

测试结果输出如下:
\begin{lstlisting}[language=c++, breaklines=true, frame=shadowbox, rulesepcolor=\color{red!20!green!20!blue!20}]

    1 2 3 4 5 
    1 2 3 4 5
    1 2 3 4 5
    one two three
    1 2 3 4 5
    1 2 3 4 5
    one two three
    4 5 6 1 2 3
    6 1 2
    0 7 6 1 2
    0 2

    -0.2 -0.1 0.1 0.2 0.3 0.4
    -0.2 -0.1 0.1 0.2 0.3 0.4 0.5 0.6
    0.1 0.2 0.3 0.4 0.5 0.6
    0.1 0.2 0.3 0.4 0.5
    1 2 3 4 5
    All tests passed!
\end{lstlisting}

测试结果一切正常,空行也正确输出。

我用 valgrind 进行测试,发现没有发生内存泄露。


\section{(可选)bug报告}

在 list.cpp 中我编写了 2 个函数,分别为 void bug1() 和 void bug2(),来复现错误。

第一个 bug 的触发条件如下:

\begin{enumerate}
    \item 首先初始化一个列表,不进行任何的赋值
    \item 然后尝试调用 list.front() 和 list.back() 打印第一个元素和最后一个元素
    \item 此时发现程序可以编译,但是无法成功运行
\end{enumerate}

据我分析,它出现的原因是:list.front() 和 list.back() 均没有对空数组进行处理,
而是直接返回了 *begin() 和 *\-\-end(). 因此我修改了代码,如下所示,这样在运行时如果遇到尝试打印空数组时会抛出错误。

\begin{lstlisting}[language=c++, breaklines=true, keywordstyle=\color{blue!70}, commentstyle=\color{red!50!green!50!blue!50}, frame=shadowbox, rulesepcolor=\color{red!20!green!20!blue!20}]
    Object &front()
    {
        if(empty())
            throw std::out_of_range("Attempting to access front of an empty list");
        return *begin(); 
    }

    Object &back()
    {
        if(empty())
            throw std::out_of_range("Attempting to access back of an empty list");
        return *--end();
    }
\end{lstlisting}

第二个 bug 的触发条件如下:

\begin{enumerate}
    \item 首先初始化一个列表
    \item 然后尝试调用 list.erase() 删除 list.end(),即哨兵节点 tail. 
    \item 此时发现程序可以编译,但是无法成功运行
\end{enumerate}

据我分析,它出现的原因是:list.erase() 没有对边界情况 head 和 tail进行处理,
而是直接删除了该节点. 因此我修改了代码,如下所示,这样就可以避免删除 head 和 tail 的情况,让程序正常运行。

\begin{lstlisting}[language=c++, breaklines=true, keywordstyle=\color{blue!70}, commentstyle=\color{red!50!green!50!blue!50}, frame=shadowbox, rulesepcolor=\color{red!20!green!20!blue!20}]
    iterator erase(iterator itr)
    {
        if(itr != tail && itr != head)     // 输入的迭代器不能是 head 和 tail
        {
            Node *p = itr.current;
            iterator retVal{ p->next };
            p->prev->next = p->next;
            p->next->prev = p->prev;
            delete p;
            theSize--;
            return retVal;
        } else if (itr == head)
            return ++itr;
        else
            return itr;
    }
\end{lstlisting}

由于我在 list.h 中修改相关设计,因此如果执行 void bug1(),会抛出错误;如果执行 void bug2(),不会有任何错误。

\end{document}

%%% Local Variables: 
%%% mode: latex
%%% TeX-master: t
%%% End: 
